\documentclass[a4paper,11pt]{article}

%use the english line for english reports
%usepackage[english]{babel}
\usepackage[portuguese]{babel}
\usepackage[utf8]{inputenc}
\usepackage{indentfirst}
\usepackage{graphicx}
\usepackage{verbatim}
\usepackage{float}
\usepackage{listings}
\usepackage{caption}
\usepackage[margin=3cm]{geometry}

\linespread{0.5}

\captionsetup{labelformat=empty,labelsep=none}
\begin{document}
\lstset{breaklines=true,
basicstyle=\ttfamily\small}
%\setlength{\textwidth}{16cm}
\setlength{\textheight}{22cm}

\title{\Huge\textbf{Protocolo de Ligação de Dados}\linebreak\linebreak\linebreak
\Large\textbf{Relatório}\linebreak\linebreak
\includegraphics[height=6cm, width=7cm]{feup.pdf}\linebreak \linebreak
\Large{Mestrado Integrado em Engenharia Informática e Computação} \linebreak \linebreak
\Large{Redes de Computadores}\linebreak
}

\author{
Jorge Filipe Monteiro Lima - 201000649
\\ Nuno Filipe Dinis Cruz - 201004232 
\\ Vasco Fernandes Gonçalves- 201006652 \\\linebreak\linebreak \\
 \\ Faculdade de Engenharia da Universidade do Porto \\ Rua Roberto Frias, s\/n, 4200-465 Porto, Portugal \linebreak\linebreak\linebreak
\linebreak\linebreak\vspace{1cm}}
%\date{Junho de 2007}
\maketitle
\thispagestyle{empty}

%************************************************************************************************
%************************************************************************************************

\newpage
\section{Sumário}
Coisas doces.

\section{Introdução}
Neste relatório iremos descrever o nosso segundo trabalho laboratorial, que teve como objectivos desenvolver uma aplicação de download FTP e a configuração de uma rede utilizando vlans em switch managed e router comercial CISCO.

\section{Aplicação de Download FTP}
Foi desenvolvida uma aplicação de download FTP com vista a estudar o protocolo e os pacotes que transitam na rede.
\subsection{Arquitectura}
Esta aplicação, cujo código se encontra em anexo, encontra-se dividida em quatro componentes principais, a saber:
\begin{itemize}
\item Interpretação do URL
\item Recepção de mensagens
\item Envio de comandos
\item Recepção do ficheiro
\end{itemize}
\vspace{5pt}

O programa recebe como argumento o URL para download, no formato
\begin{lstlisting}
	ftp://[<user>:<password>@]<host>/<url-path>
\end{lstlisting}

O URL é processado, sendo divido nas componentes \textit{user}, \textit{password}, \textit{host}, \textit{path} e \textit{filename}.Esta informação é usada para estabelecer a ligação, através de \textit{sockets} de C, com o servidor FTP, autenticar com o nome de utilizador (ou \textit{Anonymous} caso este não esteja especificado no URL), abrir a ligação em modo passivo e transferir o ficheiro.\\
O ficheiro é guardado no computador, com o mesmo nome do ficheiro original.\\

Foi implementada uma mecânica de detecção de erros, utilizando para isso o código da mensagem FTP, para assegurar que nos encontramos no passo correcto da sequência de comandos.\\

Outra melhoria foi a introdução de um timeout, configurável através de uma constante, que impede que o programa fique à espera de informação demasiado tempo.\\

Foram ainda implementados um medidor de progresso, cálculo da velocidade de transferência, indicação do tempo que esta demorou e verificação de erro na transferência (que verifica o número de bytes recebidos).\\

Para esta verificação foi usado o número de bytes do ficheiro original, que é recebido aquando da abertura do ficheiro para transferência, no \textit{socket} principal.

\pagebreak
\subsection{Exemplo de uma transferência com sucesso}
A sequência de mensagens de uma transferência com sucesso é a seguinte:
\begin{description}
\item[Estabelecer ligação] \hfill \\

220 FTP for Alf/Tom/Crazy/Pinguim
\item[user Pedro] \hfill \\

331 Please specify the password
\item[pass 123456] \hfill \\

230 Login successful
\item[pasv] \hfill \\

227 Entering Passive Mode (192,168,50,138,249,250)
\item[recv teste.txt] \hfill \\

150 Opening BINARY mode data connection for teste.txt (1632 bytes)
\end{description}

\section{Configuração da Rede e Análise }
%************************************************************************************************
\subsection{Experiência 1 - \textit{Configure an IP Network}}
\subsubsection{Objectivos}
Fazer coisas.

\subsubsection{Comandos}
\textbf{tux1}
>service networking restart
>ifconfig eth0 up
>ifcongig eth0 172.16.50.1/24

\textbf{tux4}
>service networking restart
>ifconfig eth0 up
>ifconfig eth0 172.16.50.254/24

\subsubsection{Análise de Logs}
Conforme o log em anexo, concluiem-se coisas.

%************************************************************************************************
\subsection{Experiência 2}
\subsubsection{Objectivos}

\subsubsection{Comandos}
tux2
>service networking restart
>ifconfig eth0 up
>ifcongig eth0 172.16.51.1/24

vlan 50
add-RC vlan51
>configure terminal
>interface fastethernet 0/12
>switchport mode access
>switchport access vlan 51
>end
>enable
>8nortel
>configure terminal
>vlan 50
>end
>show vlan id 50

add-tuxy1 vlany0
>configure terminal
>interface fastethernet 0/1
>switchport mode access
>switchport access vlan 50
>end

add-tuxy4 vlany0
>configure terminal
>interface fastethernet 0/4
>switchport mode access
>switchport access vlan 50
>end


vlan51
>configure terminal
>vlan 51
>end
>show vlan id 51

add-tux2 vlan51
>configure terminal
>interface fastethernet 0/2
>switchport mode access
>switchport access vlan 51

\subsubsection{Análise de Logs}

%************************************************************************************************
\subsection{Experiência 3}
\subsubsection{Objectivos}

\subsubsection{Comandos}
tux4 into a router
enable tux4 ipfwrd
tux4-eth1
>ifconfig eth1 up
>ifcongig eth1 172.16.41.253/24
>echo 1 > /proc/sys/net/ipv4/ip_forward
>echo 0 >/proc/sys/net/ipv4/icmp_echo_ignore_broadcasts

add-tux4 vlan1
>configure terminal
>interface fastethernet 0/3
>switchport mode access
>switchport access vlan 51
>end

route tuxy1 to subrede 51
route add -net 172.16.51.0/24 gw 172.16.50.254

route tux2 to tux4.254
route add -net 172.16.50.0/24 gw 172.16.51.253

\subsubsection{Análise de Logs}

%************************************************************************************************
\subsection{Experiência 4}
\subsubsection{Objectivos}

\subsubsection{Comandos}
\lstlist
configurar router (ligar consola ao router, tux4 ao switch ligar FE0/1 ao P6.1, ligar FE0/0 à porta 6
>root
>8nortel

configure router and NAT
>conf t
>ip route 172.16.50.0 255.255.255.0 172.16.51.253 
>interface fastethernet 0/0 
>ip address 172.16.51.254 255.255.255.0 
>no shutdown 
>ip nat inside 
>exit 

>interface fastethernet 0/1 
>ip address 172.16.(nosala).59 255.255.255.0 
>no shutdown 
>ip nat outside 
>exit 

>ip nat pool ovrld 172.16.(nosala).59 172.16.(nosala).59 prefix 24
>ip nat inside source list 1 pool ovrld overload 
>access-list 1 permit 172.16.50.0 0.0.0.7 
>access-list 1 permit 172.16.51.0 0.0.0.7 

>ip route 0.0.0.0 0.0.0.0 172.16.(nosala).254 
>ip route 172.16.50.0 255.255.255.0 172.16.51.253 
>end

default routes
tux4 - Rc #route add default gw 172.16.51.254
tux2 - Rc #route add default gw 172.16.51.254
tux1 - Rc #route add default gw 172.16.50.254

//opcional
del tux2 - tux4
route del -net 172.16.50.0/24

add-router vlany1
>configure terminal
>interface fastethernet 0/6
>switchport mode access
>switchport access vlan 51

\subsubsection{Análise de Logs}

%************************************************************************************************
\subsection{Experiência 5}
\subsubsection{Objectivos}

\subsubsection{Comandos}

#nano /etc/resolv.conf
search netlab.fe.up.pt
nameserver 172.16.(nosala).2

\subsubsection{Análise de Logs}

%************************************************************************************************
\subsection{Experiência 6}
\subsubsection{Objectivos}

\subsubsection{Comandos}

\subsubsection{Análise de Logs}

%************************************************************************************************
\subsection{Experiência 7}
\subsubsection{Objectivos}

\subsubsection{Comandos}

\subsubsection{Análise de Logs}

\section{Conclusões}

\section{Referências}

\section{Anexos}

\lstset{showstringspaces=false,
		frame=tb,
		caption=WireShark Exp 1 - PC 1 }

\newpage

\vspace*{\fill} 
\centering
\begin{Huge}\textbf{ANEXO - CÓDIGO FONTE}\end{Huge}
\vspace*{\fill}
\thispagestyle{empty}
\setcounter{page}{1}
\end{document}