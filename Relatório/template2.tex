\documentclass[a4paper,11pt]{article}

%use the english line for english reports
%usepackage[english]{babel}
\usepackage[portuguese]{babel}
\usepackage[utf8]{inputenc}
\usepackage{indentfirst}
\usepackage{graphicx}
\usepackage{verbatim}
\usepackage{float}
\usepackage{listings}
\usepackage{caption}
\usepackage[margin=3cm]{geometry}

\linespread{0.5}

\captionsetup{labelformat=empty,labelsep=none}
\begin{document}
\lstset{breaklines=true,
basicstyle=\ttfamily\small}
%\setlength{\textwidth}{16cm}
\setlength{\textheight}{22cm}

\title{\Huge\textbf{Redes de Computadores}\linebreak\linebreak\linebreak
\Large\textbf{Relatório}\linebreak\linebreak
\includegraphics[height=6cm, width=7cm]{feup.pdf}\linebreak \linebreak
\Large{Mestrado Integrado em Engenharia Informática e Computação} \linebreak \linebreak
}

\author{
Jorge Filipe Monteiro Lima - 201000649
\\ Nuno Filipe Dinis Cruz - 201004232 
\\ Vasco Fernandes Gonçalves- 201006652 \\\linebreak\linebreak \\
 \\ Faculdade de Engenharia da Universidade do Porto \\ Rua Roberto Frias, s\/n, 4200-465 Porto, Portugal \linebreak\linebreak\linebreak
\linebreak\linebreak\vspace{1cm}}
%\date{Junho de 2007}
\maketitle
\thispagestyle{empty}

%************************************************************************************************
%************************************************************************************************

\newpage
\section{Sumário}
Coisas doces.

\section{Introdução}
Neste relatório iremos descrever o nosso segundo trabalho laboratorial, que teve como objectivos desenvolver uma aplicação de download FTP e a configuração de uma rede utilizando vlans em switch managed e router comercial CISCO.

\section{Aplicação de Download FTP}
\subsection{Arquitectura}

\subsection{Exemplo de um download com sucesso}
Mais coisas doces. Não, docinhas.

\section{Configuração da Rede e Análise }
%************************************************************************************************
\subsection{Experiência 1}
\subsubsection{Objectivos}
\begin{itemize}
\item Configurar a rede em tux1 e tux4
\item Verificar que os PCs comunicam entre si
\end{itemize}

\subsubsection{Comandos}

\textbf{tux1}
\lstset{showstringspaces=false,
		caption=,
		frame=tb}
		
\begin{lstlisting}
>service networking restart
>ifconfig eth0 up
>ifcongig eth0 172.16.y0.1/24
\end{lstlisting}

\textbf{tux4}

\begin{lstlisting}
>service networking restart
>ifconfig eth0 up
>ifconfig eth0 172.16.y0.254/24
\end{lstlisting}


\subsubsection{Análise de Logs}
Conforme o log em anexo, concluiem-se coisas.

%************************************************************************************************
\subsection{Experiência 2}
\subsubsection{Objectivos}

\begin{itemize}
\item Criação de redes virtuais no switch
\end{itemize}

\subsubsection{Comandos}
\textbf{tux2}

\begin{lstlisting}
>service networking restart
>ifconfig eth0 up
>ifcongig eth0 172.16.y1.1/24
\end{lstlisting}

\textbf{vlan y0}

\begin{lstlisting}
>configure terminal
>interface fastethernet 0/12
>switchport mode access
>switchport access vlan y1
>end
>enable
>8nortel
>configure terminal
>vlan y0
>end
>show vlan id y0
\end{lstlisting}

\textbf{Adicionar tux1 à vlany0}

\begin{lstlisting}
>configure terminal
>interface fastethernet 0/1
>switchport mode access
>switchport access vlan y0
>end
\end{lstlisting}

\textbf{Adicionar tux4 à vlany0}

\begin{lstlisting}
>configure terminal
>interface fastethernet 0/4
>switchport mode access
>switchport access vlan y0
>end
\end{lstlisting}

\textbf{vlany1}
\begin{lstlisting}
>configure terminal
>vlan y1
>end
>show vlan id y1
\end{lstlisting}

\textbf{Adicionar tux2 à vlany1}

\begin{lstlisting}
>configure terminal
>interface fastethernet 0/2
>switchport mode access
>switchport access vlan y1
>end
\end{lstlisting}

\subsubsection{Análise de Logs}

%************************************************************************************************
\subsection{Experiência 3}
\subsubsection{Objectivos}

\begin{itemize}
\item Configurar o router
\end{itemize}

\subsubsection{Comandos}

\textbf{tux4 a servir de router}

\begin{lstlisting}
enable tuxy4 ipfwrd
tuxy4-eth1
>ifconfig eth1 up
>ifcongig eth1 172.16.41.253/24
>echo 1 > /proc/sys/net/ipv4/ip_forward
>echo 0 >/proc/sys/net/ipv4/icmp_echo_ignore_broadcasts
\end{lstlisting}

\textbf{Adicionar tux4 à vlany1}

\begin{lstlisting}
>configure terminal
>interface fastethernet 0/3
>switchport mode access
>switchport access vlan y1
>end
\end{lstlisting}

\textbf{Rotas}

\begin{lstlisting}
route add -net 172.16.y1.0/24 gw 172.16.y0.254
route add -net 172.16.y0.0/24 gw 172.16.y1.253
\end{lstlisting}

\subsubsection{Análise de Logs}

%************************************************************************************************
\subsection{Experiência 4}
\subsubsection{Objectivos}

\begin{itemize}
\item Configurar o Rc e implementar NAT
\end{itemize}

\subsubsection{Comandos}

\begin{lstlisting}
>root
>8nortel

>conf t
>ip route 172.16.y0.0 255.255.255.0 172.16.y1.253 
>interface fastethernet 0/0 
>ip address 172.16.y1.254 255.255.255.0 
>no shutdown 
>ip nat inside 
>exit 

>interface fastethernet 0/1 
>ip address 172.16.(nosala).y9 255.255.255.0 
>no shutdown 
>ip nat outside 
>exit 

>ip nat pool ovrld 172.16.1.y9 172.16.1.y9 prefix 24
>ip nat inside source list 1 pool ovrld overload 
>access-list 1 permit 172.16.y0.0 0.0.0.7 
>access-list 1 permit 172.16.y1.0 0.0.0.7 

>ip route 0.0.0.0 0.0.0.0 172.16.1.254 
>ip route 172.16.y0.0 255.255.255.0 172.16.y1.253 
>end
\end{lstlisting}

\textbf{Rotas tux4}
\begin{lstlisting}
add default gw 172.16.y1.254 
\end{lstlisting}

\textbf{Rotas tux2} 
\begin{lstlisting}
add default gw 172.16.y1.254
\end{lstlisting}

\textbf{Rotas tux1}
\begin{lstlisting}
add default gw 172.16.y0.254
\end{lstlisting}

\textbf{Rota entre tux2 e tux4}
\begin{lstlisting}
route del -net 172.16.y0.0/24
\end{lstlisting}

\textbf{Adicionar router à vlany1}

\begin{lstlisting}
>configure terminal
>interface fastethernet 0/6
>switchport mode access
>switchport access vlan y1
>end
\end{lstlisting}

\subsubsection{Análise de Logs}

%************************************************************************************************
\subsection{Experiência 5}
\subsubsection{Objectivos}

\begin{itemize}
\item Configurar o DNS
\end{itemize}

\subsubsection{Comandos}

\begin{lstlisting}
#nano /etc/resolv.conf
search netlab.fe.up.pt
nameserver 172.16.1.2
\end{lstlisting}

\subsubsection{Análise de Logs}

%************************************************************************************************
\subsection{Experiência 6}
\subsubsection{Objectivos}

\begin{itemize}
\item Conexões TCP - Testar a aplicação FTP
\end{itemize}

\subsubsection{Comandos}

\subsubsection{Análise de Logs}

\section{Conclusões}

\section{Referências}

\section{Anexos}

\newpage

\vspace*{\fill} 
\centering
\begin{Huge}\textbf{ANEXO - CÓDIGO FONTE}\end{Huge}
\vspace*{\fill}
\thispagestyle{empty}
\setcounter{page}{1}
\end{document}