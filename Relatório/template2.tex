\documentclass[a4paper,11pt]{article}

%use the english line for english reports
%usepackage[english]{babel}
\usepackage[portuguese]{babel}
\usepackage[utf8]{inputenc}
\usepackage{indentfirst}
\usepackage{graphicx}
\usepackage{verbatim}
\usepackage{float}
\usepackage{listings}
\usepackage{caption}
\usepackage[margin=3cm]{geometry}

\linespread{0.5}

\captionsetup{labelformat=empty,labelsep=none}
\begin{document}
\lstset{breaklines=true,
basicstyle=\ttfamily\small}
%\setlength{\textwidth}{16cm}
\setlength{\textheight}{22cm}

\title{\Huge\textbf{Protocolo de Ligação de Dados}\linebreak\linebreak\linebreak
\Large\textbf{Relatório}\linebreak\linebreak
\includegraphics[height=6cm, width=7cm]{feup.pdf}\linebreak \linebreak
\Large{Mestrado Integrado em Engenharia Informática e Computação} \linebreak \linebreak
\Large{Redes de Computadores}\linebreak
}

\author{
Jorge Filipe Monteiro Lima - 201000649
\\ Nuno Filipe Dinis Cruz - 201004232 
\\ Vasco Fernandes Gonçalves- 201006652 \\\linebreak\linebreak \\
 \\ Faculdade de Engenharia da Universidade do Porto \\ Rua Roberto Frias, s\/n, 4200-465 Porto, Portugal \linebreak\linebreak\linebreak
\linebreak\linebreak\vspace{1cm}}
%\date{Junho de 2007}
\maketitle
\thispagestyle{empty}

%************************************************************************************************
%************************************************************************************************

\newpage
\section{Sumário}
Coisas doces.

\section{Introdução}
Neste relatório iremos descrever o nosso segundo trabalho laboratorial, que teve como objectivos desenvolver uma aplicação de download FTP e a configuração de uma rede utilizando vlans em switch managed e router comercial CISCO.

\section{Aplicação de Download FTP}
\subsection{Arquitectura}

\subsection{Exemplo de um download com sucesso}
Mais coisas doces. Não, docinhas.

\section{Configuração da Rede e Análise }
%************************************************************************************************
\subsection{Experiência 1 - \textit{Configure an IP Network}}
\subsubsection{Objectivos}

\subsubsection{Comandos}

\subsubsection{Análise de Logs}

%************************************************************************************************
\subsection{Experiência 2}
\subsubsection{Objectivos}

\subsubsection{Comandos}

\subsubsection{Análise de Logs}

%************************************************************************************************
\subsection{Experiência 3}
\subsubsection{Objectivos}

\subsubsection{Comandos}

\subsubsection{Análise de Logs}

%************************************************************************************************
\subsection{Experiência 4}
\subsubsection{Objectivos}

\subsubsection{Comandos}

\subsubsection{Análise de Logs}

%************************************************************************************************
\subsection{Experiência 5}
\subsubsection{Objectivos}

\subsubsection{Comandos}

\subsubsection{Análise de Logs}

%************************************************************************************************
\subsection{Experiência 6}
\subsubsection{Objectivos}

\subsubsection{Comandos}

\subsubsection{Análise de Logs}

%************************************************************************************************
\subsection{Experiência 7}
\subsubsection{Objectivos}

\subsubsection{Comandos}

\subsubsection{Análise de Logs}

\section{Conclusões}

\section{Referências}

\section{Anexos}


\newpage

\vspace*{\fill} 
\centering
\begin{Huge}\textbf{ANEXO - CÓDIGO FONTE}\end{Huge}
\vspace*{\fill}
\thispagestyle{empty}
\setcounter{page}{1}
\end{document}